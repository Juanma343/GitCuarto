En este caso, se ha utilizado un vector de flotantes. Este es de tamaño  nCellsHeight * nCellsWidth, ya que de esto modo podemos tener todos los elemetos de todas las celdas, como en una matriz pero en forma de vector, de modo que si por ejemplo queremos acceder a la 3 fila y la 4 columna, teniendo 5 columnas en esta matriz, esa celda es la posicion  3 * 5 + 4 del cector, es decir, 19.

Des esta forma accedemos facilmemte en cualquier momento a cualquier celda, y si lo que queremos es de la posicion por ejemplo 21 del caso anterior, la fila de daca haciendo 21 / 5 para la fila y 21 \% 5 para columna.