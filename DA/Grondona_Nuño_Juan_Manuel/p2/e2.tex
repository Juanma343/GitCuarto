Para este caso, es necesitado de una matriz llamada tabla, y un vector llamado valor. Con esto podría ser suficiente, pero en este caso para facilitar algunas operaciones sin cambiar la estructura de la lista de defensas, he creado dos vectores auxiliares, que son coste e id. Estos vectores se encargan de guardar el coste de una defensa y de su id. De esta manera ya tenemos de todos los datos necesarios para la los siguientes algoritmos sin depender de la lista de defensa que nos paran, y todos los datos se pueden recorrer fácilmente con un simple for. Además, decir que todos los vectores anteriormente mencionados tiene tantas posiciones como defensas la lista. Por otro lado, la matriz tiene tantas filas como defensas y tantas columnas como ases menos el coste de la primera torreta, ya que esta siempre tiene que colocarse primero y no entra dentro de las valoraciones.