En lo primero en lo que se parece es en que tenemos un primer conjunto de Soluciones que serian las posibles celdas en las que colocarse. A estas se les da una valoración, y se van comprobando de una a una, de mayor a menor, como en un algoritmo devorador. Además de esto, tenemos que comprobar todas las celdas seleccionadas con la función factible, que también está en los algoritmos devoradores y se encarga de comprobar que ese caso sea válido, en nuestro caso, comprueba que se puede colocar en esa celda en concreto una torreta y que no hay nada allí que haga que colisionen o que no se salga del mapa.\\
La única diferencia de este algoritmo con respecto a los algoritmos voraces, es que ese no devuelve el conjunto solución, ya que es un problema de colocación de objetos, y se pueden colocar directamente cada objeto dentro de la función, y de este modo no tener que devolver nada.