\begin{lstlisting}
struct celda{
    int x_;
    int y_;
    int valor;
    Vector3 posicion;

    celda(int x, int y, int val, double cellW, double cellH): x_(x), y_(y), valor(val){
        posicion = Vector3(x * cellH + cellH * 0.5f, y * cellW + cellW * 0.5f, 0);
    }
};

// funcion que valorea las celdas

int cellValueBase(int row, int col, int nCellsWidth, int nCellsHeight
	, float mapWidth, float mapHeight, List<Object*> obstacles, List<Defense*> defenses) {
    
    int RadioPiedras = 13;
    double densidad = 0;

    int valor, inter;
    if (nCellsHeight <= 24 || nCellsWidth <= 24){
        inter = std::min(ceil(nCellsWidth / 3), ceil(nCellsHeight / 3));
    }
    else {
        inter = std::min(ceil(nCellsWidth / 2), ceil(nCellsHeight / 2));
    }

    int distCenWid = fabs(row - ceil(nCellsHeight / 2));
    int distCenHei = fabs(col - ceil(nCellsWidth / 2));
    valor = std::min(inter - distCenHei, inter - distCenWid);

    if (valor > 0 ){
        double distancia, media = 0;
        int maxva, aux = 0;
        for (auto it = obstacles.begin(); it != obstacles.end(); it++) {
            distancia = _distance((*it)->position, Vector3((double)row / nCellsHeight * mapHeight, (double)col / nCellsWidth * mapWidth, 0));
            if (distancia <= RadioPiedras) {
                media += distancia;
                aux++;
            }
        }
        valor += inter * media / aux;
    
    }
    return valor;
}

//funcion que considera posible esa celda

bool factible(Object* defensa, celda *cell, float mapWidth, float mapHeight, List<Object*> obstacles, List<Defense*> defenses) {
    bool res = true;
    if ((defensa->radio >= cell->posicion.x || cell->posicion.x >= mapWidth - defensa->radio) || (defensa->radio >= cell->posicion.y || cell->posicion.y >= mapHeight - defensa->radio) ) {
        res = false;
    }
    for (auto it = obstacles.begin(); it != obstacles.end() && res; it++){
        if (distObjeRad(cell->posicion, defensa->radio, (*it)->position, (*it)->radio) <= 5) {
            res = false;
        }
    }
    for (auto it = defenses.begin(); it != defenses.end() && res; it++){
        if (distObjeRad(cell->posicion, defensa->radio, (*it)->position, (*it)->radio) <= 0) {
            
            res = false;
        }
        
    }
    return res;   
}

//algoritmo devorador

void colocaBase(bool** freeCells, int nCellsWidth, int nCellsHeight, float mapWidth, float mapHeight
    , std::list<Object*> obstacles, std::list<Defense*> defenses){

    float cellWidth = mapWidth / nCellsWidth;
    float cellHeight = mapHeight / nCellsHeight; 

    std::list<celda*> valoresbase;
    for(int i = 0; i < nCellsHeight; ++i) {
        for(int j = 0; j < nCellsWidth; ++j) {
            valoresbase.push_back(new celda(i, j, cellValueBase(i, j, nCellsWidth, nCellsHeight, mapWidth, mapHeight, obstacles, defenses), cellWidth, cellHeight));
        }
    }
    valoresbase.sort([](celda* a, celda* b) -> bool{return a->valor > b->valor;});

    std::list<Defense*> posicionadas;
    auto defAct = defenses.begin();
    auto cellAct = valoresbase.begin();
    bool fin = true;
    while (cellAct != valoresbase.end() && fin){
        if (factible(*defAct, (*cellAct), mapWidth, mapHeight, obstacles, defenses)){
            (*defAct)->position = (*cellAct)->posicion;
            fin = false;
            defAct++;
        }
        else{
            cellAct++;
        }
    }

}
\end{lstlisting}