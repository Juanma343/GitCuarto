En este caso la función se divide en dos partes.
\begin{itemize}
    \item Valoracion por lo centrado que esté en el tablero.
    \item Valoracion por lo centrado que esté con respecto el número de obstáculos alrededor.
\end{itemize}
Para el primer tipo, doy una valoración positiva a la celda central, y voy disminuyendo según se aleje. La valoración más alta de las celdas es el de mínimo ente la longitud y la altitud del mapa dividida ente 3, de este modo se genera un "cuadrado" en torno a esta celda central y después de 0, otorga valores negativos.

Para la segunda parte de la valoración, comprobamos alrededor de la celda, los obstáculos que hay, después esta se valora según la media de estas rocas, cuanto mayor sea esta, mejor puntuación, ya que con un mayor media existe una mayor distancia entre estas, y de esta forma se puede posicionar bien la torreta principal.

Además, según el tamaño del mapa, el valor máximo que pueden aportas estos se ven modificados, de esta manera podemos encontrar la posición más segura que puede estar más alejada.

% Elimine los símbolos de tanto por ciento para descomentar las siguientes instrucciones e incluir una imagen en su respuesta. La mejor ubicación de la imagen será determinada por el compilador de Latex. No tiene por qué situarse a continuación en el fichero en formato pdf resultante.
%\begin{figure}
%\centering
%\includegraphics[width=0.7\linewidth]{./defenseValueCellsHead} % no es necesario especificar la extensión del archivo que contiene la imagen
%\caption{Estrategia devoradora para la mina}
%\label{fig:defenseValueCellsHead}
%\end{figure}