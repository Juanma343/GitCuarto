\documentclass[]{article}

\usepackage[left=2.00cm, right=2.00cm, top=2.00cm, bottom=2.00cm]{geometry}
\usepackage[spanish,es-noshorthands]{babel}
\usepackage[utf8]{inputenc} % para tildes y ñ
\usepackage{graphicx} % para las figuras
\usepackage{xcolor}
\usepackage{listings} % para el código fuente en c++

\lstdefinestyle{customc}{
  belowcaptionskip=1\baselineskip,
  breaklines=true,
  frame=single,
  xleftmargin=\parindent,
  language=C++,
  showstringspaces=false,
  basicstyle=\footnotesize\ttfamily,
  keywordstyle=\bfseries\color{green!40!black},
  commentstyle=\itshape\color{gray!40!gray},
  identifierstyle=\color{black},
  stringstyle=\color{orange},
}
\lstset{style=customc}

%opening
\title{Práctica 4. Exploración de grafos}
\author{Juan Manuel Grondona Nuño \\ % mantenga las dos barras al final de la línea y este comentario
juanmanuel.grondonanu@alum.uca.es \\ % mantenga las dos barras al final de la linea y este comentario
Teléfono: 656485032 \\ % mantenga las dos barras al final de la línea y este comentario
NIF: 49193526E \\ % mantenga las dos barras al final de la línea y este comentario
}


\begin{document}

\maketitle

%\begin{abstract}
%\end{abstract}

% Ejemplo de ecuación a trozos
%
%$f(i,j)=\left\{ 
%  \begin{array}{lcr}
%      i + j & si & i < j \\ % caso 1
%      i + 7 & si & i = 1 \\ % caso 2
%      2 & si & i \geq j     % caso 3
%  \end{array}
%\right.$

\begin{enumerate}
\item Comente el funcionamiento del algoritmo y describa las estructuras necesarias para llevar a cabo su implementación.

$$ f(damage, ataques por segundo, rango, dispersion, salud, coste) = $$
$$3 \cdot \frac{damage \cdot ataques por segundo}{coste} + 6 \cdot \frac{dispersion}{coste} + 6 \cdot \frac{rango}{coste} + \frac{salud}{coste} $$

Esta fórmula suma las diferentes propiedades relativas a su coste, es decir, calcula una característica como pondría ser la salud, y la divide por su coste, y de este modo podemos ver quien nos da más salud relativo al coste, y así con todas las características y después se suman. Estas características están ponderadas, es decir, que tiene un factor multiplicativo en cada una, estas están ahí para dar más importancia a una característica que a otra, de este modo tendremos la mejor torreta relación calidad precio. Esto lo calculamos de esta manera a causa de que el dinero es un factor limitante en el número posterior de torretas, por lo que es mejor usar las mejores y más baratas para usar más torretas, mejor que pocas con un poco de más daño.


\item Incluya a continuación el código fuente relevante del algoritmo.

Para la función factibilidad simplemente haremos tres observaciones. \\
\begin{itemize}
    \item La defensa no se colocara fuera del mapa.
    \item La defensa no se colocará donde coincida con algún obstáculo.
    \item La defensa no se colocará donde coincida con otra defensa.
\end{itemize}\\
Si una de estas condiciones se cumpliera, la función devolverá false, y por el contrario devolverá true.\\



\end{enumerate}

Todo el material incluido en esta memoria y en los ficheros asociados es de mi autoría o ha sido facilitado por los profesores de la asignatura. Haciendo entrega de esta práctica confirmo que he leído la normativa de la asignatura, incluido el punto que respecta al uso de material no original.

\end{document}
