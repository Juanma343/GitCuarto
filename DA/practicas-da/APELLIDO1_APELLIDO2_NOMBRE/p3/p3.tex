\documentclass[]{article}

\usepackage[left=2.00cm, right=2.00cm, top=2.00cm, bottom=2.00cm]{geometry}
\usepackage[spanish,es-noshorthands]{babel}
\usepackage[utf8]{inputenc} % para tildes y ñ
\usepackage{graphicx} % para las figuras
\usepackage{xcolor}
\usepackage{listings} % para el código fuente en c++

\lstdefinestyle{customc}{
  belowcaptionskip=1\baselineskip,
  breaklines=true,
  frame=single,
  xleftmargin=\parindent,
  language=C++,
  showstringspaces=false,
  basicstyle=\footnotesize\ttfamily,
  keywordstyle=\bfseries\color{green!40!black},
  commentstyle=\itshape\color{gray!40!gray},
  identifierstyle=\color{black},
  stringstyle=\color{orange},
}
\lstset{style=customc}


%opening
\title{Práctica 3. Divide y vencerás}
\author{Juan Manuel Grondona Nuño \\ % mantenga las dos barras al final de la línea y este comentario
juanmanuel.grondonanu@alum.uca.es \\ % mantenga las dos barras al final de la linea y este comentario
Teléfono: 656485032 \\ % mantenga las dos barras al final de la línea y este comentario
NIF: 49193526E \\ % mantenga las dos barras al final de la línea y este comentario
}


\begin{document}

\maketitle

%\begin{abstract}
%\end{abstract}

% Ejemplo de ecuación a trozos
%
%$f(i,j)=\left\{ 
%  \begin{array}{lcr}
%      i + j & si & i < j \\ % caso 1
%      i + 7 & si & i = 1 \\ % caso 2
%      2 & si & i \geq j     % caso 3
%  \end{array}
%\right.$

\begin{enumerate}
\item Describa las estructuras de datos utilizados en cada caso para la representación del terreno de batalla. 

$$ f(damage, ataques por segundo, rango, dispersion, salud, coste) = $$
$$3 \cdot \frac{damage \cdot ataques por segundo}{coste} + 6 \cdot \frac{dispersion}{coste} + 6 \cdot \frac{rango}{coste} + \frac{salud}{coste} $$

Esta fórmula suma las diferentes propiedades relativas a su coste, es decir, calcula una característica como pondría ser la salud, y la divide por su coste, y de este modo podemos ver quien nos da más salud relativo al coste, y así con todas las características y después se suman. Estas características están ponderadas, es decir, que tiene un factor multiplicativo en cada una, estas están ahí para dar más importancia a una característica que a otra, de este modo tendremos la mejor torreta relación calidad precio. Esto lo calculamos de esta manera a causa de que el dinero es un factor limitante en el número posterior de torretas, por lo que es mejor usar las mejores y más baratas para usar más torretas, mejor que pocas con un poco de más daño.


\item Implemente su propia versión del algoritmo de ordenación por fusión. Muestre a continuación el código fuente relevante. 

Para la función factibilidad simplemente haremos tres observaciones. \\
\begin{itemize}
    \item La defensa no se colocara fuera del mapa.
    \item La defensa no se colocará donde coincida con algún obstáculo.
    \item La defensa no se colocará donde coincida con otra defensa.
\end{itemize}\\
Si una de estas condiciones se cumpliera, la función devolverá false, y por el contrario devolverá true.\\



\item Implemente su propia versión del algoritmo de ordenación rápida. Muestre a continuación el código fuente relevante. 

Escriba aquí su respuesta al ejercicio 3.

\item Realice pruebas de caja negra para asegurar el correcto funcionamiento de los algoritmos de ordenación implementados en los ejercicios anteriores. Detalle a continuación el código relevante.

En lo primero en lo que se parece es en que tenemos un primer conjunto de Soluciones que serian las posibles celdas en las que colocarse. A estas se les da una valoración, y se van comprobando de una a una, de mayor a menor, como en un algoritmo devorador. Además de esto, tenemos que comprobar todas las celdas seleccionadas con la función factible, que también está en los algoritmos devoradores y se encarga de comprobar que ese caso sea válido, en nuestro caso, comprueba que se puede colocar en esa celda en concreto una torreta y que no hay nada allí que haga que colisionen o que no se salga del mapa.\\
La única diferencia de este algoritmo con respecto a los algoritmos voraces, es que ese no devuelve el conjunto solución, ya que es un problema de colocación de objetos, y se pueden colocar directamente cada objeto dentro de la función, y de este modo no tener que devolver nada.

\item Analice de forma teórica la complejidad de las diferentes versiones del algoritmo de colocación de defensas en función de la estructura de representación del terreno de batalla elegida. Comente a continuación los resultados. Suponga un terreno de batalla cuadrado en todos los casos. 

Escriba aquí su respuesta al ejercicio 5.

\item Incluya a continuación una gráfica con los resultados obtenidos. Utilice un esquema indirecto de medida (considere un error absoluto de valor 0.01 y un error relativo de valor 0.001). Es recomendable que diseñe y utilice su propio código para la medición de tiempos en lugar de usar la opción \emph{-time-placeDefenses3} del simulador. Considere en su análisis los planetas con códigos 1500, 2500, 3500,..., 10500, al menos. Puede incluir en su análisis otros planetas que considere oportunos para justificar los resultados. Muestre a continuación el código relevante utilizado para la toma de tiempos y la realización de la gráfica.

\begin{lstlisting}
struct celda{
    int x_;
    int y_;
    int valor;
    Vector3 posicion;

    celda(int x, int y, int val, double cellW, double cellH): x_(x), y_(y), valor(val){
        posicion = Vector3(x * cellH + cellH * 0.5f, y * cellW + cellW * 0.5f, 0);
    }
};

// funcion que valorea las celdas

int cellValueTorretas(int row, int col, bool** freeCells, int nCellsWidth, int nCellsHeight
	, float mapWidth, float mapHeight, List<Object*> obstacles, List<Defense*> defenses) {
    
    int valor;
    int maxVal = std::min(ceil(nCellsWidth / 3), ceil(nCellsHeight / 3));
    int distCenWid = fabs(row - defenses.front()->position.x / (mapHeight / nCellsHeight) );
    int distCenHei = fabs(col - defenses.front()->position.y / (mapWidth / nCellsWidth) );
    valor = std::min(maxVal - distCenHei, maxVal - distCenWid);
    return valor;
}

//funcion que considera posible esa celda

bool factible(Object* defensa, celda *cell, float mapWidth, float mapHeight, List<Object*> obstacles, List<Defense*> defenses) {
    bool res = true;
    if ((defensa->radio >= cell->posicion.x || cell->posicion.x >= mapWidth - defensa->radio) || (defensa->radio >= cell->posicion.y || cell->posicion.y >= mapHeight - defensa->radio) ) {
        res = false;
    }
    for (auto it = obstacles.begin(); it != obstacles.end() && res; it++){
        if (distObjeRad(cell->posicion, defensa->radio, (*it)->position, (*it)->radio) <= 5) {
            res = false;
        }
    }
    for (auto it = defenses.begin(); it != defenses.end() && res; it++){
        if (distObjeRad(cell->posicion, defensa->radio, (*it)->position, (*it)->radio) <= 0) {
            
            res = false;
        }
        
    }
    return res;   
}

//algoritmo devorador

void colocaTorre(bool** freeCells, int nCellsWidth, int nCellsHeight, float mapWidth, float mapHeight
              , std::list<Object*> obstacles, std::list<Defense*> defenses){

    float cellWidth = mapWidth / nCellsWidth;
    float cellHeight = mapHeight / nCellsHeight; 

    std::list<celda*> valorestorre;
    for(int i = 0; i < nCellsHeight; ++i) {
       for(int j = 0; j < nCellsWidth; ++j) {
           valorestorre.push_back(new celda(i, j, cellValueTorretas(i, j, freeCells, nCellsWidth, nCellsHeight, mapWidth, mapHeight, obstacles, defenses), cellWidth, cellHeight));
       }
    }
    valorestorre.sort([](celda* a, celda* b) -> bool{return a->valor > b->valor;});

    auto defAct = ++defenses.begin();
    auto cellAct1 = valorestorre.begin();
    bool fin1 = true;
    while (defAct != defenses.end()){
        while (cellAct1 != valorestorre.end() && fin1){
            if (factible(*defAct, (*cellAct1), mapWidth, mapHeight, obstacles, defenses)){
                (*defAct)->position = (*cellAct1)->posicion;
                fin1 = false;
                valorestorre.erase(cellAct1);
                cellAct1 = valorestorre.begin();
            }
            else{
                cellAct1++;
            }
        }
        fin1 = true;
        defAct++;
    }
}

void DEF_LIB_EXPORTED placeDefenses(bool** freeCells, int nCellsWidth, int nCellsHeight, float mapWidth, float mapHeight
              , std::list<Object*> obstacles, std::list<Defense*> defenses) {

    float cellWidth = mapWidth / nCellsWidth;
    float cellHeight = mapHeight / nCellsHeight; 
    
    colocaBase(freeCells, nCellsWidth, nCellsHeight, mapWidth, mapHeight, obstacles, defenses);

    colocaTorre(freeCells, nCellsWidth, nCellsHeight, mapWidth, mapHeight, obstacles, defenses);

}
\end{lstlisting}

\end{enumerate}

Todo el material incluido en esta memoria y en los ficheros asociados es de mi autoría o ha sido facilitado por los profesores de la asignatura. Haciendo entrega de este documento confirmo que he leído la normativa de la asignatura, incluido el punto que respecta al uso de material no original.

\end{document}
