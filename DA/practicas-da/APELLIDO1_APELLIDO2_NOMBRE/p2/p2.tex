\documentclass[]{article}

\usepackage[left=2.00cm, right=2.00cm, top=2.00cm, bottom=2.00cm]{geometry}
\usepackage[spanish,es-noshorthands]{babel}
\usepackage[utf8]{inputenc} % para tildes y ñ
\usepackage{graphicx} % para las figuras
\usepackage{xcolor}
\usepackage{listings} % para el código fuente en c++

\lstdefinestyle{customc}{
  belowcaptionskip=1\baselineskip,
  breaklines=true,
  frame=single,
  xleftmargin=\parindent,
  language=C++,
  showstringspaces=false,
  basicstyle=\footnotesize\ttfamily,
  keywordstyle=\bfseries\color{green!40!black},
  commentstyle=\itshape\color{gray!40!gray},
  identifierstyle=\color{black},
  stringstyle=\color{orange},
}
\lstset{style=customc}


%opening
\title{Práctica 2. Programación dinámica}
\author{Juan Manuel Grondona Nuño \\ % mantenga las dos barras al final de la línea y este comentario
juanmanuel.grondonanu@alum.uca.es \\ % mantenga las dos barras al final de la linea y este comentario
Teléfono: 656485032 \\ % mantenga las dos barras al final de la línea y este comentario
NIF: 49193526E \\ % mantenga las dos barras al final de la línea y este comentario
}


\begin{document}

\maketitle

%\begin{abstract}
%\end{abstract}

% Ejemplo de ecuación a trozos
%
%$f(i,j)=\left\{ 
%  \begin{array}{lcr}
%      i + j & si & i < j \\ % caso 1
%      i + 7 & si & i = 1 \\ % caso 2
%      2 & si & i \geq j     % caso 3
%  \end{array}
%\right.$

\begin{enumerate}
\item Formalice a continuación y describa la función que asigna un determinado valor a cada uno de los tipos de defensas.

$$ f(damage, ataques por segundo, rango, dispersion, salud, coste) = $$
$$3 \cdot \frac{damage \cdot ataques por segundo}{coste} + 6 \cdot \frac{dispersion}{coste} + 6 \cdot \frac{rango}{coste} + \frac{salud}{coste} $$

Esta fórmula suma las diferentes propiedades relativas a su coste, es decir, calcula una característica como pondría ser la salud, y la divide por su coste, y de este modo podemos ver quien nos da más salud relativo al coste, y así con todas las características y después se suman. Estas características están ponderadas, es decir, que tiene un factor multiplicativo en cada una, estas están ahí para dar más importancia a una característica que a otra, de este modo tendremos la mejor torreta relación calidad precio. Esto lo calculamos de esta manera a causa de que el dinero es un factor limitante en el número posterior de torretas, por lo que es mejor usar las mejores y más baratas para usar más torretas, mejor que pocas con un poco de más daño.


\item Describa la estructura o estructuras necesarias para representar la tabla de subproblemas resueltos.

Para la función factibilidad simplemente haremos tres observaciones. \\
\begin{itemize}
    \item La defensa no se colocara fuera del mapa.
    \item La defensa no se colocará donde coincida con algún obstáculo.
    \item La defensa no se colocará donde coincida con otra defensa.
\end{itemize}\\
Si una de estas condiciones se cumpliera, la función devolverá false, y por el contrario devolverá true.\\


\item En base a los dos ejercicios anteriores, diseñe un algoritmo que determine el máximo beneficio posible a obtener dada una combinación de defensas y \emph{ases} disponibles. Muestre a continuación el código relevante.

Escriba aquí su respuesta al ejercicio 3.

\item Diseñe un algoritmo que recupere la combinación óptima de defensas a partir del contenido de la tabla de subproblemas resueltos. Muestre a continuación el código relevante.

En lo primero en lo que se parece es en que tenemos un primer conjunto de Soluciones que serian las posibles celdas en las que colocarse. A estas se les da una valoración, y se van comprobando de una a una, de mayor a menor, como en un algoritmo devorador. Además de esto, tenemos que comprobar todas las celdas seleccionadas con la función factible, que también está en los algoritmos devoradores y se encarga de comprobar que ese caso sea válido, en nuestro caso, comprueba que se puede colocar en esa celda en concreto una torreta y que no hay nada allí que haga que colisionen o que no se salga del mapa.\\
La única diferencia de este algoritmo con respecto a los algoritmos voraces, es que ese no devuelve el conjunto solución, ya que es un problema de colocación de objetos, y se pueden colocar directamente cada objeto dentro de la función, y de este modo no tener que devolver nada.

\end{enumerate}

Todo el material incluido en esta memoria y en los ficheros asociados es de mi autoría o ha sido facilitado por los profesores de la asignatura. Haciendo entrega de este documento confirmo que he leído la normativa de la asignatura, incluido el punto que respecta al uso de material no original.

\end{document}
